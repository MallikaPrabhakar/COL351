\begin{solution}{Question 1}
    \begin{question}
        Question
    \end{question}
    \tcblower{}
    \begin{proof}[Solution]
      We propose the following algorithm for computing all-pairs-distances:
      \begin{algorithm}[H]
        \caption{Computing all-pairs-distances}\label{alg:all}
        \begin{algorithmic}[1]
          \Procedure{AllPairDistances}{$G$}
            \State{$A_G \gets adjacency(G)$}
            \State{$H \gets \text{\textsc{ComputeH}}(G)$}
            \If{$H = G$}
              \State{$D_G \gets A_G$}
              \State{$D_G \gets$ all off-diagonal zero entries are set to $\infty$}
              \State{\Return{$D_G$}}
            \EndIf{}
            \State{$D_H \gets \text{\textsc{AllPairDistances}}(H)$}
            \State{$D_G \gets \text{\textsc{ComputeDg}}(G, D_H)$}
            \State{\Return{$D_G$}}
          \EndProcedure{}
        \end{algorithmic}
      \end{algorithm}
      This is a recursive algorithm that we use to compute the all-pairs-shortest distances. To prove the same, we will prove the correctness of the algorithm using reverse induction on the depth of the recursive calls.\\

      \textbf{Base Case} If $H$ is the same as $G$, then each component in $G$ is fully connected. Therefore, the distance matrix will be the same as the adjacency matrix and the off-diagonal entries that are $0$ will be $\infty$ since there is no path between such vertices.\\
      \textbf{Inductive Step} We assume that it is true for depth $i+1$, now consider the call at depth $i$.\\
      We have already shown the correctness of line $3, 10$ in Question~\ref{ques:3a} and Question~\ref{ques:3d} respectively. Additionally from the inductive assumption, we know that $D_H$ is indeed the distance of graph $H$. Therefore, our recursive algorithm is correct.\\

      However, we still have to prove termination. To do the same, we notice that any two vertices that have a path between them have a path of length $< n$. Additionally, the distance halves at each step as proved in Question~\ref{ques:3a}. Therefore, the algorithm terminates in $O(\log{n})$ calls.\\
      \textbf{Time Complexity} As stated above, the number of calls to \texttt{AllPairDistances} is $O(\log{n})$. Each call of the function takes $O(n^\omega)$ time as shown in Question~\ref{ques:3a} and Question~\ref{ques:3d}. Therefore, the total time complexity of the algorithm is $O(n^\omega \log{n})$.\\
      This completes the algorithm along with proof of correctness and time complexity.
    \end{proof}
\end{solution}
