\begin{solution}{Question 1}\label{ques:1}
    \begin{question}
        Question
    \end{question}
    \tcblower{}
    \begin{proof}[Solution]
      Since $M$ is fairly large, we will assume that $p \approx M$ and that $M$ is a multiple of $n$. Since we have to show inequality, choosing the lower bound of $p$ works since the term only increases for larger $p$. Therefore, $rx\ \text{mod}\ p$ maps exactly to a set which is the same as $\{0, 2, \ldots, p-1 \approx M-1\}$ (from lectures 26, 27). Additionally, since we have assumed that $M$ is a multiple of $n$, we get that:
      \begin{equation}
        y\ \text{mod}\ n = \{0, 1, \ldots, n-1, 0, 1, \ldots, n-1, (M/n\ times)\ldots, 0, 1, \ldots, n-1\}
      \end{equation}
      Where $y$ is the mapping of $\{0, 2, \ldots, M-1\}$ via the function $rx \text{mod} p$. Now, each \textit{bucket} contains atleast $M/n$ elements. Therefore, taking $n$ elements from any of this bucket will give a maximum chain length of $n = \Theta(n)$. There are $n\cdot^{M/n}C_n$ such subsets of $U$. The above arguments are for $p \approx M$. If $p > M$, the number of such subsets will only increase.\\
      Therefore, we have shown that there are \textit{atleast} $^{M/n}C_n$ such subsets of $U$ of size $n$ with maximum chain length in $H_r()$ is $\Theta(n)$.
    \end{proof}
\end{solution}
