\begin{solution}{Question 2.b}
    \begin{question}
        Suppose you aim to compress a file with 16-bit characters such that the maximum character frequency is strictly less than twice the minimum character frequency. Prove that the compression obtained by Huffman encoding, in this case, is same as that of the ordinary fixed-length encoding.
    \end{question}
    \tcblower{}
    \begin{proof}
        Instead of proving the solution for a $16-bit$ encoding, we will prove this for any $k-bit$ encoding with the help of the following claim:
        \begin{claim}\label{claim:constproperty}
            For an encoding of an alphabet $A_k$ of size $2^k$ such that $f_1<2f_{2^k}$ where $F$ is the frequency vector in increasing order, $a_{2m-1}$ and $a_{2m}$ are siblings for $1\leq m\leq 2^{k-1}$ and the same inequality holds for the parents of $a_i$ ($1\leq i\leq 2^k$)
        \end{claim}
        \begin{proof}
            Let $2^k=n$. Consider the frequency vector $F$, the least two frequencies are of $a_1$ and $a_2$. Let them be merged into $b_1$ where frequency of $b_1$ is $g_1=f_1+f_2$. It is easy to see that $g_1$ is greater than all $f_i$ ($1\leq i\leq n$) using the fact that $f_n<2f_1<g_1$.\\
            We will now prove the following hypothesis using induction:
            \begin{equation}
                h(i): b_p=parent(a_{2p-1}, a_{2p}), 1\leq p\leq i\text{, and }g_p<g_q, 1\leq p<q\leq i
            \end{equation}
            \textbf{Base case:} We have shown the base case, $h(1)$ to be true above.\\
            \textbf{Induction Step:} Assume it is true for $i-1$, and consider the frequency vector after these $i-1$ merges in sorted order,
            \begin{equation}
                F=(f_{2i-1}, f_{2i}, \ldots, f_{n-1}, f_{n}, g_1, g_2, \ldots, g_{i-1})
            \end{equation}
            Thus, the least two frequencies are of $a_{2i-1}$ and $a_{2i}$ and these are merged into $b_i=parent(a_{2i-1}, a_{2i})$. Now,
            \begin{equation}
                \begin{split}
                    &g_i=f_{2i-1}+f_{2i}\\
                    \text{we know that,}&\\
                    &f_{2i-1}>f_{2(i-1)-1}\text{, and }f_{2i}>f_{2(i-1)}\\
                    \implies &f_{2i-1}+f_{2i}>f_{2(i-1)-1}+f_{2(i-1)}\\
                    \implies &g_i>g_{i-1}
                \end{split}
            \end{equation}
            Therefore, $h(i)$ is also true and this completes the induction\\
            Now, consider the statement of $h(n/2)$ ($n/2$ is an integer for $k>0$):
            \begin{equation}
                b_p=(a_{2p-1}, a_{2p}), 1\leq p\leq n/2=2^{k-1}
            \end{equation}
            This completes the proof of the first half of the claim, now consider $g_1$ and $g_{n/2}$,
            \begin{equation}
                \begin{split}
                    &2a_1>a_n\text{, from the property of the alphabet}\\
                    \implies &4a_1>2a_n\\
                    \implies &2(a_1+a_2)>a_n+a_{n-1}\\
                    \implies &2g_1>g_{n/2}
                \end{split}
            \end{equation}
            Therefore, the property (inequality) also holds for the parents of $A_k$, i.e., for $A_{k-1}=(b_1, b_2, \ldots, b_{n/2})$. This completes the proof of the claim.
        \end{proof}
        \begin{claim}\label{claim:depth}
            The depth of all nodes in the set $A_i$ ($1\leq i\leq k$) is equal to $i$ (the set $A_i$ is recursively defined as parent of $A_{i+1}$ for $1\leq i < k$)
        \end{claim}
        \begin{proof}
            We will prove the claim using induction.\\
            \textbf{Base case:} $i=1$ is true since there is a single node and its depth is $0$.\\
            \textbf{Induction Step:} Assume the hypothesis is true for $i-1$. Now, consider the children of each node in $A_{i-1}$. The depth of each child is $1$ greater than the depth of its parent. From our assumption, we know that the depth of all nodes in $A_{i-1}$ are equal to $i-1$. Therefore, the depth of each node in $A_i$ will be equal to $i$. This completes the induction and prove the claim.
        \end{proof}
        From Claim~\ref{claim:depth}, we know that the depth of all nodes in $A_k$ is equal to $k$ and hence each letter in the alphabet will be encoded using $k$ bits. This is the same as the fixed length encoding for an alphabet of size $2^k$. Therefore, we have shown that the compression obtained by Huffman encoding is the same as the ordinary fixed-length encoding and this completes the proof.
    \end{proof}
\end{solution}
