% TODO:
% no need to explain in detail about how the graph is taken as input - only mention that insertion and deletion from adj list can be done in O(1) using hashing, and O(log(n)) using a balanced-BST
% I know you just wrote a rough sketch abhi, but don't forget to modify algorithm to look more pseudocode-y (1b dekh lena for reference, or Navneel's directly)
%
% Structuring the solution post writing the algorithm:
% Termination -
% In each iteration of the loop except the last iteration, the number of vertices reduce by 1 and hence the loop will run for atmost n+1 iterations
% Correctness -
% 1. removing nodes with degree < k is necessary
% 2. removing nodes with degree > n - k - 1 is necessary
% 3. after no more such removals possible, show remaining graph satisfies the constraints
% PS: it might look noice if we replace 5 by k everywhere, gives vibes that we solved a more generic problem and we can state this at the start
%yayyyy i totally agree with all the points mentioned @sayam
\begin{solution}{Question 3.a}
    \begin{question}[]
        Present an efficient algorithm which outputs best choice of party invitees as per following specifications:\\
        1. Input: list of n people and list of pairs who know each other (undirectional)\\
        2. Every person invited should have atleast k people they know and k people they don't know at the party\\
        3. k is set to be 5 in the question but we used k for generality
    \end{question}
    \tcblower{}
    \begin{proof}[Solution]
        The main idea here is to obtain an adjacency list $adj$ and $people$ which contains information about vertices and their degrees. We keep a variable $done$ which tells us if the $people$ set is being changed or hasn't changed after a complete iteration of while loop. If there is no change i.e. $done=true$, the while loop breaks. It can also break when $people$ becomes empty. Otherwise it simply denotes that guests are still being removed to find the optimal invitee list.
        \begin{algorithm}[H]
            \caption{Generate list of invitees for 3.a}
            \begin{algorithmic}[1]
                \Procedure{Invite}{$G$}
                    \State{initialise $adj$} \Comment{empty hashmap of balanced BSTs created for adjacency list}
                    \State{initialise $people$} \Comment{empty balanced BST of degree,vertex pairs}
                    \ForAll{edge $e=\{u, v\}$ in E(G)} \Comment{sets adj}
                        \State{add v to adj[u]}
                        \State{add u to adj[v]}
                    \EndFor{}
                    \ForAll{vertex v in V(G)} \Comment{sets people}
                        \State{add \{degree(v), v\} to people}
                    \EndFor{}
                    
                    \State{$done\gets false$} \Comment{denotes if $people$ is changing}
                    \While{not $done$}
                        \State{$done\gets$ true}
                        \If{$people$ is empty}
                            \State{\textbf{break}}
                        \EndIf{}
                        \State{$\{d, v\}\gets$ vertex with minimum degree in $people$}
                        \If{$d < k$}
                            \State{$done\gets false$}
                            \State{$deletePerson(v)$}
                        \EndIf{}
                        \If{$people$ is empty}
                            \State{\textbf{break}}
                        \EndIf{}
                        \State{$\{d, v\}\gets$ vertex with maximum degree in $people$}
                        \If{$size(people)-d-1 < k$}
                            \State{$done\gets false$}
                            \State{$deletePerson(v)$}
                        \EndIf{}
                    \EndWhile{}
                    \State{initialise $invitees$} \Comment{empty array}
                    \ForAll{\{degree, person\} in $people$}
                        \State{add person to $invitees$}
                    \EndFor{}
                    \State{\Return{invitees}}
                \EndProcedure{}
            \end{algorithmic}
        \end{algorithm}
        
        The procedure $deletePerson$ takes the vertex $u$ into consideration for removal as a parameter and updates the adjacency list $adj$ and $people$. Idea behind the algorithm is to look at all the neighbours of v and remove the edge $\{u,v\}$ from $adj$ and update degree of $v$ in $people$. If degree of $v$ becomes zero, don't add it again.\\
        Since people is a balanced BST, time complexity of removal and insertion is $O(log|V|)$ because $size(people)=|V|$. Also, adjacency list can be implemented as a hashmap of balanced BSTs, so locationg a vertex $v$ takes $O(1)$ time and removing a vertex $u$ from the hashmap[v] takes $O(log \ degree)$
        \\
        Total time complexity of the procedure $deletePerson$ hence becomes
        \begin{equation}
            \begin{split}
                O(degree(u))\times O(\log|V| +\log(degree(v)))+O(\log|V| )
                &=O(degree(u)\times \log|V|)
            \end{split}
        \end{equation}

        \begin{algorithm}[H]
            \caption{sub-algorithm for $deletePerson$}
            \begin{algorithmic}[1]
                \Procedure{deletePerson}{$u$}
                \ForAll{vertex $v$ in neighbours of $u$}
                \State{remove $\{degree(v), v\}$ from $people$}
                \State{remove $u$ from $adj(v)$}
                \If{$degree(v)\neq 0$}
                    \State{insert $\{degree(v), v\}$ into $people$}
                \EndIf{}
                \EndFor
                \State{remove $\{degree(u), u\}$ from $people$}
                \State{\Return{}}
                \EndProcedure{}
            \end{algorithmic}
        \end{algorithm}
        % @TODO: the following
        \subsection*{Initialisation}
        $adj$ and $people$ are initialised as an empty hashmap and set. We then use a for loop over E(G) to set values for $adj$. Then $adj$ is used to set values for people using a for loop over V(G) as \{degree, vertex\} pairs. $done$ is set to false initially and is used as a measure to check if further changes are being made to $people$ or not.\\
        While loop works on the condition of while $done$ is not true. Inside the loop, $done$ is set to true and If $people$ turns out to be empty, it breaks. Different conditions are checked which either modify $people$ and set $done$ to false or make no change.
        
        \subsection*{Maintenance}
        For every iteration of while loop, $done$ is set to true to observe which can be changed back to false if a new change occurs.
        After checking that people is non empty, we look at the pair with minimum degree $ \{ d, v \} $.
        If the $d < k$, it implies $v$ knows less than $k$ people which is unfavourable. Hence we perform $deletePerson(*)$ operation on $v$.
        Again, we check if $people$ is now empty. If it is, then we break out of the while loop, else continue.
        Now we look at the pair with maximum degree $ \{ d, v \} $.
        If $size(people)-d-1<k$ it implies $v$ doesn't know less than $k$ people which is again, unfavourable. Hence we perform $deletePerson(*)$ operation on $v$ and move on to next iteration.
        
        \subsection*{Termination}
        The algorithm terminates if the while loop terminates. The while loop terminates if either $people$ becomes empty or done stays true i.e. no new deletePerson(*) operation was performed. The size of $people$ keeps decreasing until one of the two conditions mentioned above happens.\\
        \\
        \textbf{Case 1:} $people$ is empty\\
        There must be a $deletePerson()$ operation performed in the last iteration else the algorithm would have terminated then. Now, the size of $people$ is 0. Since there are no people to invite to the part anymore, while loop breaks based on first if condition. It terminates.
        \\
        \textbf{Case 2:} $done$ is false\\
        There was a $deletePerson()$ operation performed in the last iteration for reason same as case 1. $done$ is set to true. Now, if the minimum or maximum degree vertex was not in accordance to question statement, it will be removed and $done$ is set to false again and this will repeat until Case 1 or Case 3 is obtained since $size(people)$ will keep on decreasing after every iteration according to our logic.
        \\
        \textbf{Case 3:} $done$ is true\\
        After the $deletePerson()$ operation in last iteration, in current iteration, $done$ is set to true. It is observed that both minimum and maximum degree follow the condition specified in the question. Since no if statement is valid, $done$ is not set to false. Before the next iteration, the while loop condition is check and since $done$ was true, not $done$ is false and while loop terminates.\\
        \\
        After termination of loop, a list $invitees$ is initialised as an empty list.
        for every pair in people, the person $v$ is added to $invitees$ if people is not empty.\\
        \\
        $invitees$ is returned
        
        \subsection*{Proof of correctness}
        To prove the correctness of our algorithm, we will show the following:
        \begin{enumerate}
            \item{Removing nodes with degree $<k$ is necessary}\label{1}
            \item{Removing nodes with degree $>n-k-1$ is necessary}
            \item{No other removal is needed}
        \end{enumerate}
        The above three are necessary and sufficient to prove the correctness of our solution.
        \begin{enumerate}
            \item{Proof of \ref{1}}\\
                Consider a vertex $v$ that has a degree $<k$ at any point of time. For any final configuration (selection) of vertices from the graph, 
        \end{enumerate}
        
        \subsection*{Time complexity}
        Time complexity of $deletePerson(u)$ is $O(degree(u)\times \log|V|)$ \\
        For the algorithm,\\
        Setting $adj$ takes $O(|E|)$ time\\
        Setting $people$ takes $O(|V|)$ time\\
        Setting $invitees$ takes $O(|V|)$ time\\
        finding minimum and maximum degree vertices takes $O(\log|V|)$ time\\
        In the worst case, while loop iterates over all the vertices ($O(|V|)$ iterations)\\
        \\
        Total time complexity of the algorithm can be observed as follows:
        \begin{equation}
            \begin{split}
                &O(|E|)+O(|V|)+O(|V|)\times O(\log|V|+\sum_{u\in V}O(degree(u))\times \log|V|)+O(|V|)\\
                &=O(|V|\log|V|\timesdeg)
            \end{split}
        \end{equation}
        
    \end{proof}
\end{solution}

