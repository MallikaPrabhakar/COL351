\begin{solution}{Question 1.a}
    \begin{question}[]
        Prove that $G$ has a unique MST.\@
    \end{question}
    \tcblower{}
    \begin{proof}
        We will prove this by induction on the size of $G$ using an idea similar to Kruskal's algorithm discussed in the class.

        \textbf{Hypothesis:}
        \begin{equation}
            h(n): \forall\ G=(V,E):\ |V|=n\implies MST(G)\ is\ unique
        \end{equation}

        \textbf{Base case:} $n=1$ is true since there is no edge and $MST(G)=(V,\phi)$ is unique.

        \textbf{Induction Step:} Assume $h(n-1)$ is true for $n\geq 2$, now for $h(n)$:\\
        \textit{(Note: This proof assumes each edge to be an unordered pair of vertices)}\\

        Consider Kruskal's algorithm,
        \begin{algorithm}[H]
            \caption{Recursive MST Routine -- Kruskal's algorithm} % chktex 8
            \begin{algorithmic}[1]
                \State{$e_0\gets (x, y)$ be edge with least weight}
                \State{$H\gets G$}
                \State{remove $x, y$ from $H$ and add new vertex $z$}
                \ForAll{$v$ such that $v$ is neighbour of $x$ or $y$}
                    \State{add $(v, z)$ to $H$}
                    \State{$wt(v, z)\gets\min(wt(v, x), wt(v, y))$}
                    \If{$wt(v, x) < wt(v, y)$}
                        \State{$map(v, z)\gets (v, x)$}
                    \Else{}
                        \State{$map(v, z)\gets (v, y)$}
                    \EndIf{}
                \EndFor{}
                \State{$T_H\gets MST(H)$}
                \State{$T_G\gets (V, \{e_0\})$}
                \ForAll{$e\in T_H$}
                    \If{$e$ is not incident on $z$}
                        \State{add $e$ to $T_G$}
                    \Else{}
                        \State{add $map(e)$ to $T_G$}
                    \EndIf{}
                \EndFor{}
            \end{algorithmic}
        \end{algorithm}
        In the above algorithm, it is clear that $H$ has $n-1$ vertices. Thus, by our assumption, $h(n-1)$ is true and hence $T_H$ is unique. Also, we know that $T_G$ is a valid MST, from the correctness of Kruskal's algorithm. Now, assume by contradiction that $T_G$ is not unique. Then there exists an MST, say $T'\neq T_G$.
        \begin{claim}[]
            $e_0$ cannot be in $T'$
        \end{claim}
        \begin{proof}
            This is because, if $e_0$ were in $T'$, then $T\setminus\{e_0\}\neq T'\setminus\{e_0\}$ and thus, there would be two different MSTs for $H$ which would be a contradiction to our assumption. Thus, $e_0\notin T'$.
        \end{proof}
        Consider the path from $x$ to $y$ in $T'$. Since $e_0=(x, y)$ is not present in $T'$, there exists a different path, say $P=(f_1,f_2\cdots,f_k)$ where $f_i\in E(T'), 1\leq i \leq k$. We know that $wt(f_i) > wt(e_0), 1 \leq i \leq k$.\\
        Swap any of the $f_i$ with $e_0$ and let the subgraph formed be $T''$, i.e., $T''=T'\setminus\{f_i\}\cup\{e_0\}$. We know $T''$ is a spanning tree of $G$ since $V(T'')=V(G)$ and there are no cycles formed on performing the swap operation (this can be proven using contradiction as discussed in the lecture).\\
        Now, consider the weight of $T''$:
        \begin{equation}
            \begin{split}
                wt(T'')&=wt(T')-wt(f_i)+wt(e_0)\\
                \implies wt(T'') &< wt(T')
            \end{split}
        \end{equation}
        We have shown that the total weight of $T''$ is lesser than the weight of $T'$. However, this is a contradiction to the fact that $T'$ is the MST of $G$. Thus our assumption that $T_G$ is not the unique MST of $G$ was wrong. Therefore, $h(n)$ is true.\\

        This completes the induction and the proof that \textit{if all edge weights in a graph are distinct, then its MST is unique}.
        %        Consider the edge $e_0=(x, y)$ such that $\forall\ e\in\ E, wt(e_0) < wt(e)$ (equality cannot hold since all edges are of distinct weight).\\
        %        Now merge the two vertices $x,y$ into a single vertex $z$, and define
        %        \begin{equation}
        %            \begin{split}
        %                V'&=V\setminus \{x, y\}\cup\{z\}\\
        %                E'&=E\setminus \{e\ |\ \exists\ v\in V:\ e=(v, x)\vee e=(v, y), e\in E\} \cup \{(v, z)\ |\ \exists\ (v, x)\vee (v, y)\in E\}\\
        %                G'&=(V', E')
        %            \end{split}
        %        \end{equation}
        %        Also, for each edge $e=(v, u)\in E'$, define $wt(e)$ as:
        %        \begin{equation}
        %            wt(e)=\left\{
        %                \begin{array}{ll}
        %                    wt(e) & e\in E\\
        %                    \min(wt(v, x), wt(v, y)) & e\notin E\wedge (u=x\vee u=y)
        %                \end{array}
        %                \right.
        %        \end{equation}
        %        It is easy to see that $|V'|=n-1$ and hence from induction, we know that $MST(G')$ is unique. Call it $T'=(V', E_{T'})$. Construct $T=(V, E_T)$ from $T'$ as follows:
    \end{proof}
\end{solution}
