% --.---.---.---.---.---.---.---.---.---.---.---.---.---.---.---.---.---.---.---.---.---.--
% --.---.---.---.---.---.---.---.---.---.---.---.---.---.---.---.---.---.---.---.---.---.--
\documentclass[12pt]{article}
\usepackage[a4paper, total={6in, 9in}]{geometry}
\usepackage{graphicx}
\usepackage{algpseudocode}
\usepackage{hyperref}
\hypersetup{colorlinks=true,linkcolor=darkcyan,filecolor=darkcerulean,urlcolor=magenta}

%for proof
\usepackage{amsmath}
\usepackage{amsthm}

%for code
\usepackage{tcolorbox}
%\usepackage{minted}
%\usemintedstyle{emacs}
%\usemintedstyle{monokai}
%\inputminted{python}{file.py} to directly import code
%\inputminted{octave}{BitXorMatrix.m} single line

%for table
\usepackage{multirow}
\usepackage{multicol}

\newtheorem{theorem}{Theorem}[section]
\newtheorem{claim}[theorem]{Claim}

\definecolor{codegray}{rgb}{0.98,0.97,0.93}
\definecolor{cottoncandy}{rgb}{1.0, 0.74, 0.85}
\definecolor{darkcerulean}{rgb}{0.03, 0.27, 0.49}
\definecolor{darkcyan}{rgb}{0.0, 0.50, 0.45}
% --.---.---.---.---.---.---.---.---.---.---.---.---.---.---.---.---.---.---.---.---.---.--
% --.---.---.---.---.---.---.---.---.---.---.---.---.---.---.---.---.---.---.---.---.---.--

\title{COL351 \\ Assignment 1}
\author{Mallika Prabhakar, 2019CS50440 \\ Sayam Sethi, 2019CS10399}
\date{August 2021}
\begin{document}

% --.---.---.---.---.---.---.---.---.---.---.---.---.---.---.---.---.---.---.---.---.---.--

\maketitle
\tableofcontents
\pagenumbering{arabic}

% --.---.---.---.---.---.---.---.---.---.---.---.---.---.---.---.---.---.---.---.---.---.--
% --.---.---.---.---.---.---.---.---.---.---.---.---.---.---.---.---.---.---.---.---.---.--

\section{Question 1}

Let $G$ be an edge-weighted graph with $n$ vertices and $m$ edges satisfying the condition that all the edge weights in $G$ are distinct.
% --.---.---.---.---.---.---.---.---.---.---.---.---.---.---.---.---.---.---.---.---.---.--

\subsection{Prove that $G$ has a Unique MST}




\subsection{one point two}

% --.---.---.---.---.---.---.---.---.---.---.---.---.---.---.---.---.---.---.---.---.---.--
% --.---.---.---.---.---.---.---.---.---.---.---.---.---.---.---.---.---.---.---.---.---.--

\section{Question 2}

% --.---.---.---.---.---.---.---.---.---.---.---.---.---.---.---.---.---.---.---.---.---.--

\subsection{two point one}

% --.---.---.---.---.---.---.---.---.---.---.---.---.---.---.---.---.---.---.---.---.---.--

\subsection{two point two}

% --.---.---.---.---.---.---.---.---.---.---.---.---.---.---.---.---.---.---.---.---.---.--
% --.---.---.---.---.---.---.---.---.---.---.---.---.---.---.---.---.---.---.---.---.---.--

\section{Question 3}

% --.---.---.---.---.---.---.---.---.---.---.---.---.---.---.---.---.---.---.---.---.---.--

\subsection{three point one}

% --.---.---.---.---.---.---.---.---.---.---.---.---.---.---.---.---.---.---.---.---.---.--

\subsection{three point one}

% --.---.---.---.---.---.---.---.---.---.---.---.---.---.---.---.---.---.---.---.---.---.--
% --.---.---.---.---.---.---.---.---.---.---.---.---.---.---.---.---.---.---.---.---.---.--

\end{document}
% --.---.---.---.---.---.---.---.---.---.---.---.---.---.---.---.---.---.---.---.---.---.--
% --.---.---.---.---.---.---.---.---.---.---.---.---.---.---.---.---.---.---.---.---.---.--
