\begin{solution}{4.2}
    \begin{question}
        You are given a set of $k$ denominations. Device a polynomial time algorithm to find a change of Rs.$n$ using the minimum number of
coins.    
    \end{question}
    \tcblower{}
    \begin{proof}[Solution]
        The assumptions made are that the number of coins of every denomination are infinite and they are integral values.
        \begin{algorithm}[H]
            \caption{Find minimum number of denominations to achieve value of $n$}\label{alg:min}
            \begin{algorithmic}
            \Procedure{LeastCurr}{$denom,n$}
                \State{$k \gets size(denom)$}   \Comment{number of types of denominations}
                \State{$dpArr \gets$ array of size $n+1$ initialised with $\infty$}
                \State{$dpArr[0] \gets 0$} \Comment{Base case: no coin needed $n=0$}
                \For{$index$ in $[1, n]$}
                    \For{$i$ in $[1, k]$}
                        \If{$index-denom[i]\geq 0$}
                            \State{$dpArr[index] \gets \min(dpArr[index],dpArr[index-denom[i]]+1)$}
                        \EndIf{}
                    \EndFor{}
                \EndFor{}
                \State{\Return{$dpArr[n]$}}
            \EndProcedure{}
            \end{algorithmic}
        \end{algorithm}
    \end{proof}
    \begin{proof}[Proof of correctness]
        We will prove the correctness of Algorithm~\ref{alg:min} using induction on the value $i$.\\
        \textbf{Base case:} $i=0$ is true since we can generate a sum $0$ using $0$ coins.\\
        \textbf{Inductive step:} Assume that the claim is true for $i-1$. Consider the sum $i$\\
        The last denomination used to generate the sum $i$ has to be one of the denominations $denom[j]$ such that $i\geq denom[j]$. Therefore, the minimum number of coins needed to generate $i$ will be one more than the least number of coins needed to generate the sum $i-denom[j]$ for all valid $j$. Therefore the transition equation is given by:
        \begin{equation}
            dp(i) = 1 + \underset{j|i\geq denom[j]}\min dp(i-denom[j])
        \end{equation}
        Therefore we have shown the claim to be true for $i$ and this completes the proof of induction and hence the correctness of Algorithm~\ref{alg:min}
        \begin{proof}[Proof of termination]
            We iterate through the entire array of size n and exit successfully in any case. Hence the algorithm terminates.
        \end{proof}
        \begin{proof}[Time Complexity]
        Deciding factors for time-complexity in big-Oh notation are going through the entire $dpArray$ and running a for loop of $k$ iterations for each index.\\
        Time complexity $=O(n\times k)$\\
        This is a polynomial time solution.
        \end{proof}
        \begin{proof}[Space Complexity]
        We create a $dpArray$ of size $n+1$ and use constant space everywhere else.\\
        Space complexity $=O(n)$
        \end{proof}
    \end{proof}
\end{solution}
