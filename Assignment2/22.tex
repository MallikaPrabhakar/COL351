\begin{solution}{2.2}
    \begin{question}
        Given a set of $C$ courses, devise the most efficient algorithm to find minimum number of semesters needed to complete all $n$ courses. What is the time complexity of your algorithm? 
    \end{question}
    \tcblower{}
    \begin{proof}[Solution]
        We have courses labeled 1 to n and we have to find the minimum number of semesters taken to complete the degree requirement. We first check if the graph is a DAG or not using function $Order$ in question 2.1. If a cycle exists, we can never complete degree requirement so we $return -1$ in that scenario. Else we use a $queue$ involving $indegree$ of all courses and assign semester to all courses in the array $sem$. The maximum value in $sem$ will be the minimum number of semesters needed to complete all the courses. 
        
        \begin{algorithm}[H]
        \caption{Least number of semesters for completion}
        \begin{algorithmic}
        \Procedure{Semester}{$G$}
        \State{$existence \gets Order(G)$}\Comment{from 2.1}
        \If{$existence=[]$} \Comment{no solution case}
            \State{\Return{-1}}
        \EndIf{}
        \State{$adjList \gets generateAdjList(G)$} \Comment{Creates an adjacency list in O(V+E) time}
        \State{$indegree \gets [0 $ for i in $range(size(V)+1)]$}
        \State{$sem \gets [0 $ for i in $range(size(V)+1)]$} 
        \State{$q \gets empty\_queue()$}
        \For{$i$ in $adjList$} \Comment{calculate indegree for all vertices}
            \For{$j$ in $adjList[i]$}
                \State{$indegree[j]\gets indegree[j]+1$}
            \EndFor{}
        \EndFor{}
        
        \For{$i$ in range(1,size(V)+1)}  \Comment{initialise sem and fill the queue for indegree=0}
            \If{$indegree[i]=0$}
                \State{$sem[i]\gets 1$}
                \State{$push(q,i)$}
            \EndIf{}
        \EndFor{}
        
        \While{$size(q) \neq 0$}
            \State{$node \gets pop(q,0)$} \Comment{queue is FIFO}
            \For{$neighbour$ in $adjList[node]$}
                \State{$indegree[neighbour]\gets indegree[neighbour]-1$}
                \If{$indegree[neighbour]=0$}
                    \State{$sem[neighbour]\gets sem[node]+1$} \Comment{Pre-requisite in previous semester}
                    \State{$push(q,neighbour)$}
                \EndIf{}
            \EndFor{}
        \EndWhile{}
        \State{\Return{$\max(sem)$}}
        \EndProcedure
        \end{algorithmic}
        \end{algorithm}
    \end{proof}
    \begin{proof}[Proof of correctness]
        
        
        
        
        
    \end{proof}
    \begin{proof}[Proof of termination]
        $Semester$ function ends if either the course graph is not a directed acyclic graph since the courses can never be completed or after completion of algorithm. For the algorithm, the initial steps take finite amount of time and $for$ loops are bound by $V$. for the $while$ loop, it will run until the $queue$ becomes empty. The algorithm effectively adds all $vertices$ in the $queue$ at different times when the $indegree$ of a node turns 0 which implies there can be no duplicate additions. Since an element is popped from the queue in every iteration of while loop, while loop runs are finite and bound by $V$. Hence the algorithm terminates.
    \end{proof}
    \begin{proof}[Time Complexity]
        Following are the time complexities for different parts-
        \begin{itemize}
            \item Time to make $adjList$: $O(V+E)$ (trivial)
            \item Time to make $indegree$: $O(V)$ (trivial)
            \item Time to make $sem$: $O(V)$ (trivial)
            \item Time taken by $while$ loop: $O(V+E)$ 
            \item Time taken by $max$ function: $O(V)$ (linear)
        \end{itemize}
        Time taken by while loop is $O(V+E)$ because for termination of algorithm, it has to go through every node and it's neighbours once. Hence the overall time complexity of the $Semester$ function becomes $O(V+E)$.
    \end{proof}
    \begin{proof}[Space Complexity]
        Space complexity is determined by size of $adjList$ adjacency list, $indegree$ array, $sem$ array and $q$ queue.In worst case maximum size of queue is of $O(V)$, overall space complexity $=O(V+E)+O(V)+O(V)+O(V)=O(V+E)$
    \end{proof}
\end{solution}
