\begin{solution}{2.2}
    \begin{question}
        Given a set of $C$ courses, devise the most efficient algorithm to find minimum number of semesters needed to complete all $n$ courses. What is the time complexity of your algorithm? 
    \end{question}
    \tcblower{}
    \begin{proof}[Solution]
        We have courses labeled 1 to n and we have to find the minimum number of semesters taken to complete the degree requirement. We first check if the graph is a DAG or not using function $Order$ in question 2.1. If a cycle exists, we can never complete degree requirement so we $return -1$ in that scenario. Else we use a $queue$ involving $indegree$ of all courses and assign semester to all courses in the array $sem$. The maximum value in $sem$ will be the minimum number of semesters needed to complete all the courses. 
        
        \begin{algorithm}[H]
        \caption{Least number of semesters for completion}
        \begin{algorithmic}
        \Procedure{Semester}{$G$}
        \State{$existence \gets Order(G)$}\Comment{from 2.1}
        \If{$existence=[]$} \Comment{no solution case}
            \State{\Return{-1}}
        \EndIf{}
        \State{$adjList \gets generateAdjList(G)$} \Comment{Creates an adjacency list in O(V+E) time}
        \State{$indegree \gets [0 $ for i in $range(size(V)+1)]$}
        \State{$sem \gets [0 $ for i in $range(size(V)+1)]$} 
        \State{$q \gets empty\_queue()$}
        \For{$i$ in $adjList$} \Comment{calculate indegree for all vertices}
            \For{$j$ in $adjList[i]$}
                \State{$indegree[j]\gets indegree[j]+1$}
            \EndFor{}
        \EndFor{}
        
        \For{$i$ in range(1,size(V)+1)}  \Comment{initialise sem and fill the queue for indegree=0}
            \If{$indegree[i]=0$}
                \State{$sem[i]\gets 1$}
                \State{$push(q,i)$}
            \EndIf{}
        \EndFor{}
        
        \While{$size(q) \neq 0$}
            \State{$node \gets pop(q,0)$} \Comment{queue is FIFO}
            \For{$neighbour$ in $adjList[node]$}
                \State{$indegree[neighbour]\gets indegree[neighbour]-1$}
                \If{$indegree[neighbour]=0$}
                    \State{$sem[neighbour]\gets sem[node]+1$} \Comment{Pre-requisite in previous semester}
                    \State{$push(q,neighbour)$}
                \EndIf{}
            \EndFor{}
        \EndWhile{}
        \State{\Return{$\max(sem)$}}
        \EndProcedure
        \end{algorithmic}
        \end{algorithm}
    \end{proof}
    \begin{proof}[Proof of correctness]\\
        \textbf{Case 1:} Graph has a cycle\\
        If the graph of courses has a cycle, it implies that the courses can never be completed since the prerequisite of a course is dependent on the course. To check cycles in the graph, we use an O(V+E) time algorithm $Order$ made in question 2.1 which return an empty array $[]$ if graph contains a cycle since a topological order would not be possible. If the result of function $Order(G)$ is $[]$, we return -1 denoting completion of all courses is impossible.\\
        \\
        \textbf{Case 2:} Solution exists\\
        Here, the invariant is that after every iteration of while loop, $\max(curr\_iter)\leq \max(previous\_iter) + 1$, i.e., the maximum number of semesters needed to complete the courses until the courses \textit{visited} until current index is not more than $1+$ maximum number of semesters needed for courses until the previous iteration\\
        \textbf{Proof by induction}\\
        Base case: The base case is true since in the first iteration, we update $sem[node]$ by 1 which will be $<2$ since all nodes in the queue had $sem[node]=1$.\\
        Inductive step: Assume that the claim is true for $(i-1)\textsuperscript{th}$ iteration, consider the $i\textsuperscript{th}$ iteration. We notice that the nodes present in the queue have a value which is one of $\max(previous\_iter)$ or $\max(previous\_iter)-1$. Therefore, if the node that is updated is added in the queue, it will have a value $\leq\max(previous\_iter)+1$. Therefore, we have shown the inductive claim to be true for the $i\textsuperscript{th}$ iteration.\\
        This completes the proof of induction and hence the invariant of the while loop.\\
        We now show that each course is done as early as possible. From the claim of induction, we have shown that the maximum semesters needed until the $i\textsuperscript{th}$ iteration does not exceed $\max(previous\_iter) + 1$. Therefore, the new course is done in the same semester as $\max(previous\_iter)$ or in the next semester. In the case when it is done in the next semester, it could not have been done in an earlier semester since the indegree would have been $>0$.\\
        Therefore we have shown that our algorithm returns the most optimal solution.

    \end{proof}
    \begin{proof}[Proof of termination]
        $Semester$ function ends if either the course graph is not a directed acyclic graph since the courses can never be completed or after completion of algorithm. For the algorithm, the initial steps take finite amount of time and $for$ loops are bound by $V$. for the $while$ loop, it will run until the $queue$ becomes empty. The algorithm effectively adds all $vertices$ in the $queue$ at different times when the $indegree$ of a node turns 0 which implies there can be no duplicate additions. Since an element is popped from the queue in every iteration of while loop, while loop runs are finite and bound by $V$. Hence the algorithm terminates.
    \end{proof}
    \begin{proof}[Time Complexity]
        Following are the time complexities for different parts-
        \begin{itemize}
            \item Time to make $adjList$: $O(V+E)$ (trivial)
            \item Time to make $indegree$: $O(V)$ (trivial)
            \item Time to make $sem$: $O(V)$ (trivial)
            \item Time taken by $while$ loop: $O(V+E)$ 
            \item Time taken by $max$ function: $O(V)$ (linear)
        \end{itemize}
        Time taken by while loop is $O(V+E)$ because for termination of algorithm, it has to go through every node and it's neighbours once. Hence the overall time complexity of the $Semester$ function becomes $O(V+E)$.
    \end{proof}
    \begin{proof}[Space Complexity]
        Space complexity is determined by size of $adjList$ adjacency list, $indegree$ array, $sem$ array and $q$ queue.In worst case maximum size of queue is of $O(V)$, overall space complexity $=O(V+E)+O(V)+O(V)+O(V)=O(V+E)$
    \end{proof}
\end{solution}
