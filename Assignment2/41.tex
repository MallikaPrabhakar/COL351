\begin{solution}{4.1}
    \begin{question}
        You are given a set of $k$ denominations. Devise a polynomial time algorithm to count the number of ways to make change for Rs.$n$, given an infinite amount of coins/notes of denominations, $d[1], \ldots, d[k]$.
    \end{question}
    \tcblower{}

    \begin{proof}[Solution]
        The assumptions made are that the number of coins of every denomination are infinite and they are integral values.\\
        We solved this problem using dynamic programming. Given the cost $n$ and array of possible denominations $denom$ with size $k$, we create $dpTable$ which is an $(n+1)$ array. $dpTable[i]$ counts the number of ways to generate value $i$ using the given denominations.
        The answer is obtained by observing value of the last element $dpTable[n]$.
    \begin{algorithm}[H]
        \caption{Find total possible combinations of denominations to achieve value of $n$}\label{alg:combis}
        \begin{algorithmic}
        \Procedure{Combinations}{$denom,n$}
            \State{$k \gets size(denom)$}   \Comment{number of types of denominations}
            \State{$dpTable \gets $1D-zero array of size $(n+1)$}
            \State{$dpTable[0]\gets 1$} \Comment{there is trivially one way to generate sum $0$}
            \For{$j$ in $[1,k]$}
                \For{$i$ in $[denom[j],n]$}
                    \State{$dpTable[i] \gets dpTable[i]+dpTable[i-denom[j]]$}
                \EndFor{}
            \EndFor{}
            \State{\Return{$dpTable[n][k]$}}
        \EndProcedure{}
        \end{algorithmic}
    \end{algorithm}
        \begin{proof}[Proof of correctness]
            We will prove the correctness of the transitions in $dpTable$ using induction on the denominations used.\\
            \textbf{Base case:} $j=0$ is true since there is trivially one way to generate sum $0$ using no denominations and no other value can be generated.\\
            \textbf{Inductive step:} Assume the claim is true for $j-1$, consider denomination $j$:
            We have generated the number of ways we can generate all values using the first $j-1$ denominations. Now, consider the $j\textsuperscript{th}$ denomination.\\
            We perform another induction on the values $i$. The base case for $i=denom[j]-1$ is trivially true (there is no way to generate the sum using the $j\textsuperscript{th}$ denomination). Assume that we have generated all valid combinations for the first $i-1$ values using the first $j$ denominations. Now, we can generate value $i$ using the first $j-1$ denominations (the value currently in $dpTable[i]$) or we can use the first $j$ denominations. To use the $j\textsuperscript{th}$
            denomination, we need to use $denom[j]$ atleast once. Therefore, we need to find the number of ways we can generate $i-denom[j]$ using the first $j$ denominations. This is already computed by our inductive assumption. Therefore, the number of ways to generate $i$ using the first $j$ denominations is given by:
            \begin{equation}
                dp(i) = dp(i) + dp(i - denom[j])
            \end{equation}
            This completes the induction on the value $i$.\\
            We have shown that number of ways to generate all values is done correctly in the $j\textsuperscript{th}$ step. Therefore, we have proved the correctness of the inductive claim for $j$.
            This completes the induction and proof of correctness for Algorithm~\ref{alg:combis}.
        \end{proof}
        \begin{proof}[Proof of termination]
            Here, we have a finite table of size $(n+1)$. We iterate through the entire table and exit successfully in any case. Hence the algorithm terminates.
        \end{proof}
        \begin{proof}[Time Complexity]
        Deciding factors for time-complexity in big-Oh notation are going through the entire $dpTable$ and running a for loop with $k$ iterations at each index of the table.
        Time complexity $=O(n\times k)$\\
        This is a polynomial time solution.
        \end{proof}
        \begin{proof}[Space Complexity]
        We create a $dpTable$ of size $n+1$ and use constant space everywhere else.\\
        Space complexity $=O(n)$
        \end{proof}
    \end{proof}
\end{solution}
