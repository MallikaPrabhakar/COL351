\begin{solution}{2.1}
    \begin{question}
        Given a set of C courses, devise the most efficient algorithm to find out an order for taking the courses so that a student is able to take all the n courses with the prerequisite criteria being satisfied, if such an order exists. What is the time complexity of your algorithm?
    \end{question}
    \tcblower{}
    
    \begin{proof}[Solution]
        In this problem, an order of completing courses exists if there is no loop in the graph. A directed graph ($G$) is generated by setting all courses as vertices($V or C$) and the pre-requirements of a course as edges($E$) which is directed from the pre-requisite course to the course which needs it as pre-requisite. We use topological sort for obtaining a solution and return an empty array if no order exists(i.e. there is a loop in the graph).
    \begin{algorithm}[H]
        \caption{The way in which courses can be completed}
        \begin{algorithmic}
        \Procedure{Order}{$G$}
            \State{$adjList \gets generateAdjList(G)$} \Comment{Creates an adjacency list in O(V+E) time}
            \State{$colour \gets new \ dictionary()$}
            \State{$colour[node] \gets 0$ for all nodes} \Comment{0 represents unvisited}
            \State{$tOrder \gets []$}
            \State{$valid \gets true$}
            \ForAll{node in $G$}
                \If{$colour[node]=0$}   \Comment{node is unvisited}
                    \State{perform $DFS(node,adjList,colour,valid)$} \Comment{directly alters variables at source}
                \EndIf{}
            \EndFor{}
            \If{$valid=true$}
                \State{\Return{$reverse(tOrder)$}}\Comment{tOrder should be reversed}
            \Else
            \State{\Return{$[]$}}
            \EndIf{}
        \EndProcedure{}
        \end{algorithmic}
    \end{algorithm}
    
    \begin{algorithm}[H]
        \caption{DFS function}
        \begin{algorithmic}
        \Procedure{DFS}{$node,adjList,colour,valid$}
            \State{$colour[node]=1$}
            \ForAll{$neighbours$ in $adjList[node]$}
                \If{$colour[neighbour]=0$} \Comment{neighbour is also unvisited}
                \State{$DFS(neighbour,adjList,colour,valid)$}
                \ElsIf{$colour[neighbour]=1$} \Comment{Implies cycle exists}
                \State{$valid\gets false$}
                \State{\Return{}}
                \EndIf{}
            \EndFor{}
            \State{$color[node]\gets2$} \Comment{node is fully processed}
            \State{push $node$ in $tOrder$} \Comment{add node at the end}
        \EndProcedure{}
        \end{algorithmic}
    \end{algorithm}
    \end{proof}
    \begin{proof}[Proof of correctness]
        %@TODO: correctness is boring
    \end{proof}
    \begin{proof}[Proof of termination]
        To check if the algorithm terminates, as for loop has finite number of steps, we need to check the termination of $DFS$ function. The $for$ loop in DFS terminates if-
        \begin{enumerate}
            \item There is a cycle
            \item All the $neighbours$ are processed
        \end{enumerate}
        Since the number of $neighbours$ are finite, Condition 2 also terminates after finite steps.
        Moreover, the number of times DFS function in the $Order$ function is called can not be larger than |$V$|. Hence the algorithm terminates.
    \end{proof}
    \begin{proof}[Time Complexity]
    Following are the methods to be considered for analysind time complexity-
        \begin{itemize}
            \item Time to make adjList: $O(V+E)$ (trivial)
            \item Time to make colour: $O(V)$ (trivial)
            \item Time taken by main $for$ loop of $Order$:
            $O(V+E)$ (since DFS)
        \end{itemize}
        Hence the overall time complexity of the $Order$ function becomes $O(V+E)$
    \end{proof}
    \begin{proof}[Space Complexity]
        Space complexity is determined by size of $adjList$ adjacency list and $colour$ array.
        Complexity=$O(V+E)+O(V)$=$O(V+E)$
    \end{proof}
    
\end{solution}
